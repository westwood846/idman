% Motivation
One major obstacle when mining developer-information from software repositories is the management of their identities. Often one physical person uses multiple aliases or email-addresses when interacting with a online community. This is especially true for GitHub or Git in general, where author-metadata for a commit is generated from the users local configuration, which can be changed arbitrarily and may even vary across the machines of one person. Thus, in order to mine all the data of one physical person, all of these ``artifacts'' must be managed and mapped to one identity.

% What we did
This paper makes the following contributions: We analyze the need for such an \emph{identity management} by examining six repositories on GitHub and comparing how many identities exist for a naive versus a managed approach. We then show that existing identity management solutions are overkill for GitHub and can actually harm the result by merging too many artifacts. Finally, we present a slimmed down approach that generates nearly perfect results for all six examined repositories.

% Structure
The remainder of this paper is structured as follows: Section 2 describes related work on the topic of identity management. Section 3 documents our work in detail. Section 4 validates our approach, and finally section 5 discusses our results. Additionally, there is an appendix with all the generated data and graphics.
